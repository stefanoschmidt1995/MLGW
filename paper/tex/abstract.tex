\begin{abstract}
We apply machine learning methods to build a time-domain model for
gravitational waveforms from binary black hole mergers, called \texttt{mlgw}.
The dimensionality of the problem is handled by representing the 
waveform's amplitude and phase using a principal component analysis.
We train \texttt{mlgw} on about $\mathcal{O}(10^3)$ \texttt{TEOBResumS} and \texttt{SEOBNRv4} effective-one-body
waveforms with mass ratios $q\in[1,20]$ and aligned dimensionless
spins $s\in[-0.80,0.95]$. The resulting models are faithful to the
training sets at the ${\sim}10^{-3}$ level (averaged on the parameter space).
The speed up for a single waveform generation is  a factor 10 to 50 (depending 
on the binary mass and initial frequency) for \texttt{TEOBResumS} and
approximately an order of magnitude more for \texttt{SEOBNRv4}.
Furthermore, \texttt{mlgw} provides a closed form expression for the waveform 
and its gradient with respect to the orbital parameters; such an information might 
be useful for future improvements in GW data analysis. 
As demonstration of the capabilities of \texttt{mlgw} to perform a
full parameter estimation, we re-analyze the public data from the first
GW transient catalog (GWTC-1). We find broadly consistent results with previous 
analyses at a fraction of the cost, although the analysis with spin aligned waveforms 
gives systematic larger values of the effective spins with respect to previous analyses 
with precessing waveforms. Since the generation time does not depend 
on the length of the signal, our model is particularly suitable for the analysis 
of the long signals that are expected to be detected by third-generation detectors. 
Future applications include the analysis of waveform systematics and
model selection in parameter estimation.

\end{abstract}

