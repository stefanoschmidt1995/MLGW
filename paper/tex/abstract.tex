\begin{abstract}
We apply Machine Learning methods to build a model designed to generate a gravitational waveform in the time domain as produced by a binary black hole coalescence. 
Our model matches the accuracy of the state-of-the-art EOB models and has a tiny generation cost, equivalent to those of a fast Reduced Order Model.
As the generation time does not depend on the lenght of the signal, the model is particularly suitable for long observation surveys, such as ET.
Furthermore, it provides a closed form expression for the waveform and its gradient with respect to the orbital parameters. This could lead to a further improvement of the sampling algorithms employed for the parameter estimation.
\par
We train our model on a number of waveforms computed by \texttt{TEOBResumS} and we infer a relation between the waveform and the masses $m_1$, $m_2$ and (aligned) spins $s_1$, $s_2$ of the two BHs. 
%We reduce the dimensionality of our problem by decomposing each waveform in amplitude and phase and it is further represented in a lower dimensional space using a Principal Component Analysis. The regression from orbital parameters to principal components is performed using a Mixture of Expert model.
Our implementation is publicly available as a Python package \texttt{mlgw} as \url{https://pypi.org/project/mlgw/}.
\texttt{mlgw} has all the features required for performing a full parameter estimation (including the waveform dependence on geometrical parameters). 
We employ \texttt{mlgw} to analyse the public data from GWTC-1, the first GW transient catalog. Our results are largely compatible with those published by the LIGO-Virgo collaboration.

\end{abstract}

